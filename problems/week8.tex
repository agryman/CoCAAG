\documentclass[11pt]{amsart}


\usepackage{etoolbox}
\newtoggle{answers}
\togglefalse{answers}

% Comment out the following line to not show answers.
\toggletrue{answers}




\usepackage{amssymb,amsmath,amsfonts,amsthm,mathrsfs,fullpage}
%,amsmath,amsfonts,amsthm,mathrsfs}
\usepackage[all]{xy}  % for commutative diagrams
\SelectTips{cm}{10}
\usepackage{mathrsfs,float}
\numberwithin{equation}{section}

\usepackage{tikz}
\usetikzlibrary{arrows,shapes,positioning,calc,patterns,cd}


% Color comments
\usepackage{color}

%\definecolor{backgroundcolor}{rgb}{1,1,0.8}
%\pagecolor{backgroundcolor}

\newcommand{\commentr}[1]{{\color{red} [#1]}}
\newcommand{\commentb}[1]{{\color{blue} [#1]}}
\newcommand{\commentm}[1]{{\color{magenta} [#1]}}

\newcommand{\bnum}{\begin{enumerate}} %Use with \item to create numbered lists
\newcommand{\enum}{\end{enumerate}}
\newcommand{\babc}{\bnum\renewcommand{\labelenumi}{(\alph{enumi})}}%Use with \item for abc lists
\newcommand{\eabc}{\end{enumerate}}
\newcommand{\bijk}{\bnum\renewcommand{\labelenmi}{\roman{enumi}.}}%Roman numeral lists
\newcommand{\eijk}{\end{enumerate}}

\usepackage{verbatim,moreverb}

\newcommand{\kk}{\mathbb k}
\renewcommand{\AA}{\mathbb A}
\newcommand{\BB}{\mathbb B}
\newcommand{\CC}{\mathbb C}
\newcommand{\DD}{\mathbb D}
\newcommand{\EE}{\mathbb E}
\newcommand{\FF}{\mathbb F}
\newcommand{\GG}{\mathbb G}
\newcommand{\HH}{\mathbb H}
\newcommand{\II}{\mathbb I}
\newcommand{\JJ}{\mathbb J}
\newcommand{\KK}{\mathbb K}
\newcommand{\LL}{\mathbb L}
\newcommand{\MM}{\mathbb M}
\newcommand{\NN}{\mathbb N}
\newcommand{\PP}{\mathbb P}
\newcommand{\QQ}{\mathbb Q}
\newcommand{\RR}{\mathbb R}
\renewcommand{\SS}{\mathbb S}
\newcommand{\TT}{\mathbb T}
\newcommand{\UU}{\mathbb U}
\newcommand{\VV}{\mathbb V}
\newcommand{\WW}{\mathbb W}
\newcommand{\XX}{\mathbb X}
\newcommand{\YY}{\mathbb Y}
\newcommand{\ZZ}{\mathbb Z} 

\newcommand{\Zhat}{\widehat\ZZ}

\newcommand{\calA}{\mathcal A} 
\newcommand{\C}{\mathcal C} 

\newcommand{\OO}{\mathcal O}
\newcommand{\F}{\mathscr F} %\renewcommand{\P}{\mathscr P}
\newcommand{\Hh}{\mathscr H} 
\newcommand{\N}{\mathscr N}\newcommand{\Ii}{\mathscr I}
\newcommand{\Z}{\mathscr Z}
\newcommand{\bb}{\mathfrak b}\newcommand{\m}{\mathfrak m}\newcommand{\M}{\mathcal M}
\newcommand{\aA}{\mathfrak a} \newcommand{\fF}{\mathfrak f}
\newcommand{\qq}{\mathfrak q} 
\newcommand{\p}{\mathfrak p} \newcommand{\Pp}{\mathfrak P}
\newcommand{\norm}[1]{ \left|\left| #1 \right|\right|  }
\newcommand{\ang}[1]{ \langle #1 \rangle  }
\newcommand{\aside}[1]{ \marginpar{#1} }
\newcommand\legendre[2]{\Bigl(\frac{#1}{#2}\Bigr) }   \newcommand\Angle[2]{\langle #1,#2 \rangle}


\def\Spec{\operatorname{Spec}} \def\id{\operatorname{id}}
\def\Div{\operatorname{Div}}\def\tr{\operatorname{tr}}
\def\Supp{\operatorname{Supp}} \def\Gal{\operatorname{Gal}}

\def \GL {\operatorname{GL}_2}  
\def \PGL {\operatorname{PGL}_2}
\def \SL {\operatorname{SL}_2}
\def \PSL {\operatorname{PSL}_2}

\def\Res{\operatorname{Res}}
\def\Aut{\operatorname{Aut}} \def\End{\operatorname{End}}

\def\Prim{\operatorname{Prim}} \def\Fr{\operatorname{Frob}}
\def\lcm{\operatorname{lcm}} \def\Li{\operatorname{Li}}
\newcommand{\Hom}{\operatorname{Hom}}
\newcommand{\Card}{\operatorname{Card}}
\newcommand{\coker}{\operatorname{coker}}
\newcommand{\coimage}{\operatorname{coimage}}
\newcommand{\image}{\operatorname{image}}
\newcommand{\Ext}{\operatorname{Ext}}
\def \Ev {\operatorname{Ev}}
\def \rad {\operatorname{rad}}

\def \G {\mathcal G}
\def \B {\mathcal B}





\def\power{\mathcal{P}}
\def\Xx{\mathscr X} \def\Zz{\mathscr Z} \def\Ss{\mathscr S} \def\Gg{\mathscr G}
\def\Rr{\mathscr R}\def\Dd{\mathcal D}
 \def\cC{\mathfrak c}
\def\join{\vee}
\def\meet{\wedge}
\def\ord{\operatorname{ord}}

\def\tors{\operatorname{tors}}
\def\sfree{\operatorname{sf}}


\def\bbar#1{\setbox0=\hbox{$#1$}\dimen0=.2\ht0 \kern\dimen0 \overline{\kern-\dimen0 #1}}
\newcommand{\Qbar}{{\overline{\mathbb Q}}} 
\newcommand{\Kbar}{\bbar{K}} 
\newcommand{\kbar}{\bbar{k}} 
\newcommand{\Fbar}{\bbar{F}} 
\newcommand{\FFbar}{\overline{\FF}} 

\newcommand{\smallpmod}[1]{\text{ }(\operatorname{mod } #1 )}

\newcommand{\defi}[1]{\textsf{#1}} % for defined terms

\newtheorem{thm}{Theorem}[section]
\newtheorem{lemma}[thm]{Lemma}
\newtheorem{cor}[thm]{Corollary}
\newtheorem{prop}[thm]{Proposition}
\newtheorem{conj}[thm]{Conjecture} 
\newtheorem{example}[thm]{Example}
\newtheorem{claim}{Claim}

\theoremstyle{definition}
\newtheorem{definition}[thm]{Definition}

\theoremstyle{remark}
\newtheorem{remark}[thm]{Remark}
\newtheorem{remarks}[thm]{Remarks}


\newenvironment{romanenum}{\hfill \begin{enumerate} \renewcommand{\theenumi}{\roman{enumi}}}{\end{enumerate}}
\renewcommand{\Re}{\operatorname{Re}}

\definecolor{webbrown}{rgb}{.6,0,0}
\usepackage[
   %    draft,
        colorlinks,
        linkcolor=webbrown,  filecolor=webcolor,  citecolor=webbrown, 
        backref,
        %pdfauthor={David Zywina}, % add other authors
    %   pdftitle={Paper title goes here},
]{hyperref}
\usepackage[alphabetic,backrefs,lite]{amsrefs} % for bibliography
%Uncomment to double space:
%\renewcommand{\baselinestretch}{2}

\newcounter{excount}
\newcommand\exercise[1]{\addtocounter{excount}{1}\noindent\textbf{Problem \arabic{excount}: }#1{$ $}\\}

\newcommand\answer[1]{
\iftoggle{answers}{{\noindent \color{blue} #1\\ }}{ } 
}


\begin{document}
\setcounter{excount}{0}

\title{CoCAAG, Winter 2025: Week 8 Questions}
\maketitle
\vspace{0.5cm}

\bigskip

\exercise{{\bf Edge ideals}
  \babc
\item Write a function in Macaulay2 that, given a graph $G$ on $n$ vertices $0, 1, \ldots, n-1$, creates the
  corresponding ideal generated by $x_i x_j$, for each edge $(i,j)$ in $G$.
\item Take a specific example, and compute its primary decomposition, minimal primes, and Alexander dual.
  \item What is the definition of a vertex cover of a graph?
\item For a general edge ideal $I_G$, describe its Alexander dual (i.e. what is it?).
  \eabc
}    
{\tiny\begin{verbatim}
  needsPackage "NautyGraphs"
  strs = generateGraphs(6, MinDegree => 2)
  Gs = strs/stringToGraph
  G = Gs_5
  edges G
  (edges G)/toList/sort
  (edges G)/toList/sort//sort
\end{verbatim}
}

\bigskip

\exercise{{\bf Algebraically independent sets}
  Let $I \subset R = K[x_1, \ldots, x_n]$ be an ideal.
  We say that a subset $u \subset x = \{x_1, \ldots, x_n\}$ is
  an {\it (algebraically) independent set} if $I \cap K[u] = \{0\}$.
  \babc
\item Show that the set of independent sets is a simplicial complex.
\item Show that the independent sets
  of $I$ and of $\sqrt{I}$ are the same.
\item Now let $I$ be a monomial ideal.  Choose one, and find this complex.
\item Make a conjecture based on this example (and possibly others!) as to what this complex is
  for a general (square-free) monomial ideal.
  \item Prove this conjecture!
  \eabc
}


\medskip
 
\exercise{{\bf Tree ideals}
  For each positive integer $n \ge 2$, consider the following monomial ideal in $\mathbb{K}[x_1, \ldots, x_n]$.
  \[I_n := \big\langle \big( \prod_{x \in F} x \big)^{n - |F| + 1} \mid
    \emptyset \ne F \subset \{ x_1, \ldots, x_n \} \big\rangle
    \]
    \babc
  \item Write a Macaulay2 function to create this ideal (and ring).
  \item For $n=2$ or $n=3$, try computing a primary decomposition ``by hand'', and also the Alexander dual,
    and check your work with Macaulay2
  \item why are these called tree ideals?  (look at standard monomials in the ideal, is there a relationship with labelled
    trees on $n+1$ vertices?)
    \eabc
}
{\tiny\begin{verbatim}
R = QQ[a,b,c,d,e]
I = monomialIdeal(a*b, b*c, c*d^3, d*e, a*b*c^2)
irreducibleDecomposition I
primaryDecomposition I
dual I
\end{verbatim}
}




\end{document}
